%! TEX root = thesis.tex

\chapter{Conclusion}%
\label{sec:conclusion}

We have shown that the study successfully improves the baseline models that are representative of the current work in this domain. In Section \ref{sec:purpose}, we presented three research questions for which we want to get answers. We leveraged the DALI dataset and ran experiments that used techniques such as Singing Voice Separation (SVS), Transfer Learning, Beam Search decoding with N-gram language modeling and encoder-decoder architectures. In terms of the research questions, we have successfully showed that data augmentation techniques such as Music Source Separation can improve performance of Songs Lyrics Transcription models. Additionally, the study presented showcases evidence that Domain Adaptation through Transfer Learning and Language Models along with Beam Search Decoding can improve performance for the Songs Lyrics Transcription problem. In addition to the above, we aimed to replicate the success of encoder-decoder models in the Song Lyrics space in the presence of low data by using pre-trained models as encoder and decoder. However, through our study, we were not able to improve our best model but was still able to perform better than the initial baseline model (Wav2vec2 and Whisper). Through domain adaptation using improved language model based decoding and transfer learning, we have improved the baselines by 60\% (0.36 WER improvement) for the Wav2Vec2 based architectures and 57.4\% (0.31 WER) for the Whisper architecture.


In terms of Future Work, there are two areas that show promise to solve the problem. One area of research is to leverage more datasets for training the models. Studies \cite{radford2023robust} \cite{xu2021self} \cite{ou2022transfer} show a significant improvement in Word Error Rates (WER) when increasing the data size. The second area of future work is to utilize transformer language models as decoders by treating them as text correction modules. The hypothesis is that we can leverage the full benefit of the Acoustic model as well as the transformer language model as they will be able to utilize their existing knowledge that they have been pre-trained on. We hope to continue the research with the goal of bringin Songs to Lyrics Transcription on par with the Automatic Speech Recognition (ASR) and have sub 5\% WER.