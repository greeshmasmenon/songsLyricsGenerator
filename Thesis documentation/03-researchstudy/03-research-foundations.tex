%! TEX root = thesis.tex

\chapter{Research Study}%
\label{sec:researchstudy}


\section{Problem Formulation}%
\label{sec:problemformulation}

\section{Evaluation Methodology}%
\label{sec:evaluationmetric}

\section{Data}%
\label{sec:data}


\subsection{DALI Dataset}
\label{dalidataset}


\subsection{Performing Singing Voice Separation (SVS) using DEMUCS}%
\label{sec:svsusingdemucs}


\subsection{Data Augmentations Techniques}%
\label{sec:dataaugmentations}

\section{Research Studies}%
\label{sec:experiments}

\subsection{Baseline Model}%
\label{sec:baseline}

\subsection{Zero shot Evaluation}%
\label{sec:svsusingdemucs}

\subsubsection{Whisper Benchmark}%
\label{sec:whisperbenchmark}

\subsubsection{Wav2vec2 Benchmark}%
\label{sec:whisperbenchmark}

\subsection{Fine-tuning Pre-trained Wav2Vec2 Model with Songs}%
\label{sec:finetunepretrainedwav2vec2}

\subsection{Fine-tuning Trained Wav2Vec2 ASR Models with Songs}%
\label{sec:finetunewav2vec2}

\subsection{Wav2Vec2 + Seq2Seq Models Learning with Songs}%
\label{sec:seqtoseqlearning}

\subsection{Whisper Finetuning}%
\label{sec:whisperfinetuning}

\subsection{Introducing Songs Language Model}%
\label{sec:songlanguagemodel}
* This could be things like BART finetuned on DALI dataset. Interesting later work.
Developed architecture / system design / implementation: 1/3

\begin{itemize}
    \item start with a theoretical approach
    \item describe the developed system/algorithm/method from a high-level point of view
    \item go ahead in presenting your developments in more detail
\end{itemize}
