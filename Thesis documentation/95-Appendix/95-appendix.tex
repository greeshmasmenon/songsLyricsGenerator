%! TEX root = thesis.tex

\chapter{Appendix}%
\label{sec:elements_of_music}

\section{Elements of Music}
Music is the combination of vocal with instrument sounds to produce harmony, expression, emotion or beauty. This is something enjoyed by both the musicians and the listeners throughout the world. as the art of combining vocal or instrumental sounds (or both) to produce expression of emotion and harmony. This is an interesting definition and definitely aids the experience felt by people listening to music all over the world. However, from a mathematics or experts view, we can also categorize music to be termed as “organized sound”, making this definition more alluring from a mathematics/computer science perspective. Therefore, the goal of this particular summary is to understand what music is and the subsequent elements that music can be broken down into. Music can be broken down into the following elements \cite{martineau2021elements}


\renewcommand{\arraystretch}{2}
\centering
\setlength{\arrayrulewidth}{0.3mm}
\begin{table}[H]
\small
\begin{center}
\begin{tabular}{ | p{2cm}| p{11cm}| }
\multicolumn{2}{c}{Elements of Music \cite{martineau2021elements}} \\
\cline{1-2}
\textbf{Element of Music} & \textbf{Meaning} \\
\hline  \hline
Rhythm          &  Rhythm is the element of "TIME" in music. Typically, it is when follow the beats of the music and tap your foot to what is being played. Tempo, Duration and Meter some important aspects of this element.  \\
\hline
Dynamics        &  The quality associated with relative loudness of a musical instru-
ment or music is considered as dynamics. It is categorized into six levels -  fortissimo, forte, mezzo-forte, mezzo-piano, piano, pianissimo in increasing loudness.  Also, increasing loudness (crescendo) and decreasing loudness (descendo) are features of dynamics  \\
\hline
Melody          & Linear or Horizontal presentation of pitch. This typically refers to the highness or lowness of musical sound. Melodies are classified as - Conjunct or Disjunct depending on the difficulty of playing the sound \\
\hline
Harmony         &        This is the verticalization of pitch. It is classified as  Dissonance  (harsh sounding) or Consonance (pleasant sounding) \\
\hline
Timbre          &         This is the quality of the sound or voice. For example, the guitar ‘C’ and vocal ‘C’ are the same but sound very different because of Timbre.                                \\
\hline
Texture        &       This is the quality that talks about the number of melodies and their relationship with each other. Types include  Monophonic, Homophonic, Polyphonic or Imitative  \\
 \hline  \hline
\end{tabular}
\caption{\label{elements-of-music} \textbf{Elements of Music}}
\end{center}
\end{table}
