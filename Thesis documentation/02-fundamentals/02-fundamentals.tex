%! TEX root = thesis.tex

\chapter{Related Work and Foundational Theory}%
\label{sec:fundamentals}

Fundamentals / environment and related work: 1/3

\section{Acoustic Modelling}%
\label{sec:acousticmodelling}

\subsection{Spectrograms}%
\label{sec:spectrograms}

\subsection{Mel Frequency Cepstral Coefficients (MFCC)}%
\label{sec:mfcc}

\subsection{Hidden Markov Models(HMM)}%
\label{sec:hmm}

\section{Automatic Speech Recognition(ASR)}%
\label{sec:automaticspeechrecognition}

\subsection{E2E ASR Systems}%
\label{sec:e2easrsystems}

\subsection{Language Models (LM)}%
\label{sec:languagemodels}

\subsection{Wav2Vec2.0 - Self Supervised Approach for Speech Recognition}%
\label{sec:wav2vec2}

\subsection{Whisper - Weak Supervised Learning for Speech Recognition}%
\label{sec:whisper}

\section{Music Information Retrieval (MIR)}%
\label{sec:musicinformationretrieval}

\subsection{Singing Voice Separation (SVS) using DEMUCS}%
\label{sec:demucs}

\section{Discussion}%
\label{sec:foundationaltheorydiscussion}

Automatic Speech Recognition is a field that started with the Hidden Markov Models (HMM) that are trained using Gaussian Mixture Models (GMM). With the advent of Deep Learning models, new architectures have replaced the existing legacy ones to help calculate better ways to train and convert acoustic elements such as speech into text. In the songs to lyrics generator, we will go through these architectures and see how they are done so that we could learn how the ASR methodologies work and inspire us to do the same in the songs to lyrics conversion. The following are some of the approaches that are currently state of the art techniques that are in this particular area:

\begin{itemize}
    \item comment on employed hardware and software
    \item describe methods and techniques that build the basis of your work
    \item review related work(!)

    
\end{itemize}
